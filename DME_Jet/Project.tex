% Thesis Proposal

\documentclass[11pt,english]{article}
\usepackage{graphicx}
\usepackage[T1]{fontenc}
\usepackage[latin9]{inputenc}
\usepackage{float}
\usepackage{color}
\usepackage{amsmath,array}
\usepackage{caption}
\usepackage{units}
\usepackage{verbatim}
\usepackage{geometry,subcaption}
\geometry{verbose,tmargin=1in,bmargin=1in,lmargin=1in,rmargin=1in}
\begin{document}

\title{Data Processing Manual}

\author{Sili Deng}
\maketitle

\section{Ensight Profiles}
\begin{enumerate}
  \item Get Plot3D files from statistics.
  \item Set max/min, white background, and linear/logrithmic.
  \item \textbf{View} Uncheck everything in Axis triad visibility.  Check Bound visibility.  Uncheck Perspective. 
  \item \textbf{Viewports} Edit.  Change the background to white.  \textbf{Bounds} General is set to be 2D.  Chang the color correspondingly.  Change the Axis specific by adjusting the Axis origin and Width.  For X, origin is $0.25$, and width is $0.15$.  For Y, origin is $0.25$, and width is $0.40$. 
  \item Use Transformation Editor to adjust the viewports display.
  \item Save two sets of figures.  The first one is with Extent location BOTH and Gradation ON to record the dimensions.  The second one is without Gradation.
  \item When \textbf{Export} png files, watch out the resolution of the graphic card settings and the export image dimensions ($1166 * 755$).   
  \item \textbf{Export} Scenario for later use.  However, query marks will not be able to display when reload the scenario file.
  \item Use PowerPoint to do the post-processing.  Create black frames with tics and figure out the tics.  Be sure to add a white background for the final version, otherwise, component margins appear to be a black line in Latex for png files.
\end{enumerate}

\section{CEMA}
\begin{enumerate}
  \item Create mixture fraction iso-contours.
  \item Create queries and be sure they are on nodes.
  \item Choose up to ten quantities and save them to a file.
  \item Save a png to keep track of query locations.
  \item Combine data files to get \emph{input.mat}.  It should be a p by n matrix, where p is the number of sample points and n number of species plus temperature.  The order of the column should be the same as the CHEMKIN chemical mechanism \emph{chem.bin}.
  \item Copy \emph{input.mat} to a folder that includes both \emph{chem.bin} and \emph{CEMA\_SD.m}.  Be sure to run the MATLAB file on a WIN32 machine.  Run the file by copying the first line of the function.
  \item CEMA generates three solution files: \emph{mat\_r.mat}, \emph{mat\_s.mat}, and \emph{em\_mat}.  \item Run \emph{generate\_CEMA\_out.m} within the folder that contains \emph{List\_s.mat} and \emph{List\_r.mat}.
  \item The CEMA solution file \emph{CEMA.out} is generated in the destination folder.  
\end{enumerate}

\section{Lagrangian Flamelet Analysis}
\begin{enumerate}
  \item Create \emph{CA.in} file.
  \begin{enumerate}
    \item Get profiles from Ensight, including the displacement velocity.
    \begin{enumerate}
      \item Vector $\nabla\chi$.  Create Grad(CHI), and name it GC.
      \item Scalar $\|\nabla\chi\|$.  Create SQRT(DOT(GC,GC)), and name it MGC.
      \item Vector $\frac{\nabla\chi}{\|\nabla\chi\|}$.  Create GC/MGC, and name it n.
      \item Vector $\rho D \nabla\chi$.  Create DIFF*GC, and name it I.
      \item Scalar $\nabla \cdot (\rho D \nabla\chi)$.  Create Div(I), and name it DI.
      \item Vector $-\frac{\nabla \cdot (\rho D \nabla\chi)}{\rho \|\nabla\chi\|}$.  Create -DI/$\rho$/MGC*n, and name it VD.
      \item Scalar $U + VD[x]$.  Create U+VD[x], and name it RU.
      \item Save the Scenerio file for later use.
    \end{enumerate}
    \item On Zst iso-contour, sample CHI and RU, and save as chi\_x\_Zst and u\_x\_Zst, respectively.
    \item Run \emph{getCA\_in.m} to get \emph{CA.in}.
    \item Extrapolate CHI to avoid the influence of ignition, which causes increasing CHI.
    \begin{enumerate}
      \item Load \emph{CA.in} into MATLAB.
      \item Visualize plot(log(CA(:,1)),log(CA(:,6))), and pick the start and end (en) points to give a nice linear decreasing CHI v.s. time plot.
      \item Use Tool -> Basic statistics to get linear fit coefficients, and save those coefficients as matrix for reference.
      \item "y = exp(fit.coeff(1) * log(CA(en+1:end,1)) + fit.coeff(2));".
      \item "newCHI = [CA(1:en,6);y];".
      \item Check the plot, and save newCHI.mat.  "plot(CA(:,1),CA(:,6)); hold on; plot(CA(:,1),newCHI);". 
      \item Copy \emph{CA.in} to \emph{CA.fit}, and replace CHI with newCHI.\\
      For some cases, we need to run LFA longer to get the inhomogeneous autoignition response.  To do that, we have to extrapolate time, distance, and CHI.
      \item Load newCHI.mat and CA.
      \item "I = find(CA(:,6)-newCHI);" take down the index as start.
      \item "plot(log(CA(start:end,1)),log(newCHI(start:end)));"
      \item Save linear regression coefficients.
      \item Create a new time array and extrapolate CHI:
        \begin{itemize}
          \item "st = 2*CA(end,1) - CA(end-1,1);"
          \item "dif = CA(end,1) - CA(end-1,1);"
          \item "ext\_t = linspace(st,st+dif*4499,4500)';"
        \end{itemize}
      \item Save new t, x, and CHI as ext.mat:
        \begin{itemize}
          \item "ext\_CHI = exp(fit.coeff(1) * log(ext\_t) + fit.coeff(2));"
          \item "ext = [ext\_t ext\_t*3.2 ext\_CHI];"
        \end{itemize}
      \item Create CA.ext file.
    \end{enumerate}
  \end{enumerate}
  \item Create \emph{TimeZero.FM} for initial conditions.
  \begin{enumerate}
    \item Clip x at ten times of the wall thickness ($0.0004$ m).  
    \item Sample Z(y), T(Z), species(Z).  Be sure to use the same species names as the chemical mechanism.
    \item Check Z file and clip extraneous zeros and ones.  Take down the start and end indices.
    \item Run \emph{StartFile.m} to get \emph{TimeZero.FM}.
  \end{enumerate}  
  \item Run LFM.  Check I/O files.  If \emph{CA.in} has more than 5000 times, FM will stop at the 5000th line.  Need to continue the calculation by providing the start file and pay attention to the time.
  \item Process 1D LFM solution files.
  \begin{enumerate}
    \item Create a symbolic file \emph{syms} with "time" in it.
    \item "LT -s syms -l Z,<mixture fraction> -r <output file> <data file>".
    \item Sort the output file.  "sort -s -n -k 1,1 <output file> > <new file>".
  \end{enumerate} 
  \item Process 2D statistical files.
  \begin{enumerate}
    \item Sample U (RU), T, HO2, and H2O2 along mixture fraction iso-contours.
    \item Run \emph{int\_time.m} to get the species time history files \emph{species\_time.out}.
  \end{enumerate} 
\end{enumerate}

\end{document}
