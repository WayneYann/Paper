\abstract{Chemical kinetics and fluid dynamics are crucial components of combustion, governing the efficiency, stability, and emissions of many practical combustion devices.  Particularly, this dissertation advances the understanding of the coupling effects between chemical kinetics and transport in flame dynamics (Chapters~\ref{ch:NTC} and~\ref{ch:dynamics}) and soot emissions (Chapters~\ref{ch:biofuel} and~\ref{ch:bluff}) at engine relevant conditions.  For both topics, foundational studies on chemical kinetics were first carried out in relatively simple, laminar, low-dimensional configurations with well characterized flow fields to understand low-temperature cool flame chemistry and soot chemistry.  Complexities from flows were then considered, and chemistry-transport coupling was investigated at engine relevant conditions to elucidate the role of low-temperature chemistry in autoignition-affected flame dynamics and the role of hydrogen addition in soot evolution in bluff body flames, leveraging the understanding obtained in the chemical kinetics studies.  

The first half of this dissertation focuses on low-temperature chemistry and its role in flame dynamics.  Specifically, in Chapter~\ref{ch:NTC}, experimental studies, supported by computations, were conducted on the coupling of low-temperature chemistry and transport in the ignition, extinction, and associated steady burning in nonpremixed DME/air counterflow flames.  The presence of low-temperature chemical reactivity was detected nonintrusively, and the ignition temperature was determined subsequently.  At elevated pressures, which promote low-temperature chemistry, the hysteresis in ignition and extinction behavior of nonpremixed cool flames was observed and quantified for the first time.  The thermal and chemical structure of the cool flame was computationally analyzed to elucidate the dominant chemical pathways during the ignition and extinction processes.  Effects of strain rate, fuel and oxygen concentration, and ambient pressure on the cool flame were investigated.  Possible reasons for the discrepancies between experiments and computations were discussed to facilitate further cool flame studies and the development of low-temperature chemical models.

The role of low-temperature chemistry in autoignition-affected flame dynamics was then computationally investigated in Chapter~\ref{ch:dynamics}.  Laminar nonpremixed DME/air coflow flames were investigated at elevated temperatures and pressures with various boundary temperatures and velocities.  The stabilization mechanism for steady flames and the flame dynamics for the forced oscillating cases were analyzed.  Besides the tribrachial structure typically observed at nonautoignitive conditions, a multibrachial thermal structure was observed due to autoignition.  Consequently, a stabilization regime diagram was proposed, including frozen flow, kinetically stabilized (autoignition), autoignition-propagation-coupled stabilized, kinematically stabilized (tribrachial flame), and burner stabilized regimes.  The transition of the combustion mode was elucidated through the computational investigations of sinusoidally forced oscillating cases.  Transition between a multibrachial autoignition front and a tribrachial flame occurs periodically and exhibited a hysteresis.  First-stage low-temperature chemistry is less affected by flow dynamics with only second-stage autoignition and flame chemistry, which accounts for the majority of the heat release, coupled with flow oscillation.  The understanding of the role of low-temperature chemistry in flame dynamics under laminar autoignitive conditions lays the foundation for future studies at turbulent conditions in practical engines. 

The second half of this dissertation focuses on soot emissions.  To understand the fuel effects on soot chemistry, in Chapter~\ref{ch:biofuel}, the sooting limits of nonpremixed $n$-heptane, $n$-butanol, and methyl butanoate flames were determined experimentally in a liquid pool stagnation-flow configuration.  In addition, complementary simulations with detailed polycyclic aromatic hydrocarbon (PAH) chemistry and a detailed soot model, based on the Hybrid Method of Moments (HMOM), were performed and compared with the experimental critical strain rates for the sooting flames.  Argon dilution was used to keep the thermal environment for the three fuel cases nearly the same to elucidate the chemical effects.  Both experiment and simulation showed that $n$-heptane and $n$-butanol had similar sooting characteristics, while methyl butanoate had the least sooting propensity.  Further sensitivity and reaction pathway analysis demonstrated that the three fuels share similar PAH chemical pathways, and C$_5$ and C$_6$ ring formation from the intermediate chain species was found to be the rate-limiting step.  The differences in sooting propensity were due to the role of fuel bounded oxygen and the fuel breakdown processes.  The findings in this chapter provide guidance to the design of diesel/biofuel blendings to reduce soot emissions.

Finally, in Chapter~\ref{ch:bluff}, the evolution of soot in a turbulent nonpremixed bluff body ethylene/hydrogen flame was investigated using a combination of experiments and Large Eddy Simulations and compared with a neat ethylene counterpart.  With hydrogen addition, the maximum soot volume fractions in the recirculation zone and jet-like region significantly decreased.  Flamelet calculations demonstrated that hydrogen addition suppressed soot formation due to the reduction of the C/H ratio, resulting in an estimated fourfold reduction in soot volume fraction due to chemical effects.  Soot reduction in the downstream jet-like region of the flame was quantitatively consistent with this chemical effect.  However, soot reduction in the recirculation zone was substantially larger than this analysis suggests, indicating an additional hydrodynamic effect.  Large Eddy Simulation was used to further investigate soot evolution in the recirculation zone and to elucidate the role of hydrogen addition.  For the same heat release rate and similar jet Reynolds number as the neat ethylene case, the addition of hydrogen required a higher jet velocity, and this led to a leaner recirculation zone that inhibited soot formation and promoted soot oxidation.  The findings in this chapter further validated the comprehensive soot model for turbulent sooting flames and advanced the understanding of soot evolution in recirculating flows.} 

