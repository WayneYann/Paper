\chapter{Introduction\label{ch:intro}}

\section{Recombination X-ray}
It is as early as the 1970's, the gain of recombination scheme of
X-ray lasers is predicted, by scientist from Soviet Union. But this
scheme is verified experimentally until 1985, by Szymon Suckewer,
who observed X-ray radiation at the wavelength of $18.2nm$, from
Carbon 3->2 transition. Then he made another breakthrough at 1993, in
Lithium, where the lasing between 2->1 is realized. The signification
is that it is the first experiment achieved lasing to ground state.

Recent news that X-ray free electron laser at SLAC (Stanford) became
operational at $0.15$ nm provided tremendous excitement for scientists in
various disciplines. Very large efforts made by many researchers, engineers
and technicians over a dozen of years at a very substantial cost is finally
paying off. The very large cost of such a large \textquotedblleft scientific
tool\textquotedblright\ did not diminish the importance of the achievement,
which is expecting to have a profound impact on the studies in many fields
such as crystallography, and condense matter, in general, and high
resolution microscopy of biological elements, etc. in particular. However,
very large cost and size, resulting in low flexibility regarding the use of
this important device has forced researchers to search for a more portable,
less expensive XUV and X-ray devices, which can be used in individual
researchers laboratories for preparatory work before going into these larger
facilities.

In particular, a portable X-ray laser operating in a transient regime at $%
13.5$ nm has been demonstrated in Suckewer's group \cite{Koro96} (see Fig. %
\ref{Li}). The laser uses H-like Li ions as an active medium which are
excited by ionization-recombination process in a microcapillary. The
excitation mechanism is the following. First, a strong $250$ $fs$ laser
pulse (power density 10$^{17}$W/cm$^{2}$) creates nonequilibrium plasma of Li%
$^{3+}$ ions and electrons. Electron density is $10^{19}-10^{20}$ cm$^{-3}$.
The ion-electron recombination occurs on a time scale faster then $1$ ps
which creates Li$^{2+}$ (H-like) ions in highly excited states (large
principle quantum number $n$). By collisions atoms are transferred to states
with smaller $n$ on a few ps time scale. For proper density population
inversion between the level $n=2$ and the ground state ($n=1$) is created
for $10-100$ ps. This results in lasing on the wavelength of the $n=2$ to $%
n=1$ transition ($13.5nm$) which is measured by a spectrometer. Experimental
setup and the measured spectrum is shown in the figure \ref{Li}.

As demonstrated in the figure \ref{Li}, a low power $2$ Hz Nd/YAG laser ($100
$ mJ$,$ $5ns$) was focused with an $f/6$ lens on the entrance of
microcapillary. Microcapillaries of lengths from $1$ to $5$ $mm$ and
diameters of $0.3$ $mm$ were made in solid LiF by drilling appropriate
holes. After a few hundred nsec delay (varied between 300 and 1000 ns), the
Ti:sapphire $250$ $fs$ laser was fired, whose energy is $50-60$ mJ in a
repetition rate $2$ Hz. This femto-second laser was tightly focused with the
same $f/6$ lens onto the plasma at the entrance of the microcapillary
providing a power density close to $2\times 10^{17}$ $W/{cm}^{2}$. The
Nd/YAG laser beam was directed to the microcapillary by a near
100reflectivity mirror, $M_{1}$, and the femto-second laser beam was
directed by directed by a mirror, $M_{2}$ which is transparent for the $1.06$
$\mu m$ wavelength of the Nd/YAG laser. From the output spectra, we could
see the soft X-ray lasing.

In the present research we are trying to make a laser in a similar way,
however, instead of ions, we want to use neutral He atoms which are easier
to confine then plasma. Fig. \ref{He} shows He energy level diagram and
radiative decay times for various transitions. We want to make a laser
operating at the $2~^{1}$P $\rightarrow 1~^{1}$S transition ($58$ nm). So,
we need to create population inversion between those levels. We are planning
to do it in two stages. First, we populate the triplet $2~^{3}$S
state by ionization-recombination processes similar to what has been done in
the Li-ions experiment. Then we transfer population into the singlet $2~^{1}$
P state of He by sending short optical laser pulses which yields population
transfer via the triplet $3~^{3}$D state.

\subsection{Preparation of orthohelium}
First we inject an ultrashort high power laser pulse to ionize the He gas.
Then we turned off the laser and rapid recombination and de-excitation
follow such that the lowest states of He atoms are prepared according to
their statistical weights. Hence for the sake of simplicity we take the
relative population of the triplet $2~^{3}\text{%S}$ and singlet $1~^{1}\text{%
  S}$ states to be 3 to 1.

Let me consider the physics of the laser plasma as produced in our Princeton
lab. We envision a laser plasma created by Keldysh tunneling with a non
Boltzmann distribution of neutral excited atoms. This involves $\text{He}%
^{++}\rightarrow \text{He}^{+}\rightarrow \text{He}$ electron capture via
three-body recombination. Three-body recombination for H-like ions is
approximately proportional to the forth power of the principal quantum
number $n^{4}$ and to the square of the electron density as ${N_{e}}^{2}$.
Hence for sufficiently high initial electron density three-body
recombination will dominate radiative decay.

However, the collisional ionization from highly excited states is also fast,
thus in order for three-body recombination rates to dominate ionization
rates, the recombining plasma should have a low electron temperature $\text{T%
}_{e}$. For example, for $k_{B}\text{T}_{e}\approx 1$ $eV$, initial electron
density $N_{e}\approx 2\times 10^{16}$ cm$^{-3}\approx N_{\text{He}^{++}}$
for fully ionized $\text{He}^{++}$ in quasi-neutral plasma, the equilibrium
between recombination and ionization processes will be established after a
few dozen picoseconds at a ratio of $N_{e}/N_{\text{He}}\approx 3\times
10^{-3}$ for $N_{\text{He}}=10^{16}$ cm$^{-3}$. This ratio becomes even
smaller ($N_{e}/N_{\text{He}}\approx 10^{-8}$) for $k_{B}\text{T}_{e}\approx
0.5$ $eV$. In such way one can obtain an essentially neutral gas of excited
He atoms.

In order to create a fully ionized $\text{He}^{++}$ plasma at low
temperature, we consider the example of a plasma capillary 10-100 $\mu $m in
diameter and a few cm long. The tunneling ionization can be used to generate
the plasma \cite{Keld65,Burn89,avi04,avi07}. In this way one can strip both
electrons from He atoms without significantly heating the plasma, especially
for ultra-short laser pulses. The laser intensity needs to be in the order
of $10^{16}$ W/cm$^{2}$ for efficient tunneling ionization of He to $\text{He%
}^{+}$ and to $\text{He}^{++}$ according to Keldysh theory \cite{Keld65}.
For needle like plasma column such intensities can easily be obtained from a
Ti/Sapphire laser at wavelength $\lambda =0.8$ $\mu $m with $\sim $ $10$ mJ
energy per pulse in pulses of $50-100$ fs duration with ionization pulse
propagating in plasma channel. Use of such short pulses is crucial to
minimize plasma heating. It is important to use laser pulses shorter than
collision times of electrons in order to minimize plasma heating during the
ionization process.

The bottom line is that we can create a cold laser plasma which recombines
to produce an excited neutral gas. In particular the metastable triplet $%
2~^{3}\text{S}$ (8000 s radiative life time) state will be formed with a
statistical weight of around 3 compared to the $1^{1}\text{S}$ state. The
population in the $2~^{3}\text{S}$ state is then transferred to the $3~^{3}%
\text{D}$ state via the $2~^{3}\text{P}$ levels by dark state adiabatic
transfer (STIRAP).

\subsection{Population transfer from $2~^{3}\text{S}$ to $3~^{3}$D via STIRAP
  process}

Robust population transfer from the triplet $2~^{3}\text{S}$ to triplet $%
3~^{3}\text{D}$ is then made possible by Stimulated Raman Adiabatic Passage
(STIRAP) \cite{Berg98}. In this technique one subjects the system, whose
state is $2~^{3}\text{S}$ at $t=0$, to a so called counter intuitive pulse
sequence with Rabi frequencies $\Omega _{1}$ and $\Omega _{2}$ in which the $%
\Omega _{1}$ ($2~^{3}\text{P}\rightarrow 3~^{3}\text{D}$) pulse precedes the
$\Omega _{2}$ ($2~^{3}\text{S}\rightarrow 2~^{3}\text{P}$) pulse. This pulse
sequence ideally results in a complete transfer of population to the desired
state $3~^{3}\text{D}$ without necessarily populating the $2~^{3}\text{P}$
state in the process.

The mechanism of STIRAP is best understood in the dressed state basis in
which we introduce bright and dark states. Beginning with the dark state
\begin{equation}
  |0~\rangle =\frac{\Omega _{1}|2~^{3}\text{S}\rangle -\Omega _{2}|3~^{3}\text{%
      D}\rangle }{\sqrt{\Omega _{1}^{2}+\Omega _{2}^{2}}}
\end{equation}%
we apply $\Omega _{1}$ before $\Omega _{2}$ so that $|0\rangle \cong |2^{3}%
\text{S}\rangle $ during the early stages of transfer. Then we adiabatically
turn on $\Omega _{2}$ while turning off $\Omega _{1}$, such that $|0\rangle
\cong |3~^{3}\text{D}\rangle $ for large times. The condition of
adiabaticity implies the following estimate of the required pulse energy
(see Appendix A)

\begin{equation}
  W\gtrsim 1000\frac{\hbar cS}{\lambda ^{3}\gamma \tau _{\text{pulse}}},
  \label{s3a}
\end{equation}%
where $S$ is the cross section area of the pulse, $\tau _{\text{pulse}}$ is
the pulse duration, $\lambda $ and $\gamma $ are the wavelength and the rate
of the transition. For the weakest 2$^{3}$P$\rightarrow $2$^{3}$S transition
$\lambda =1083$ nm and $\gamma =10^{7}$s$^{-1}$. Then for a plasma capillary
of radius $\sim 0.1$mm and pulse duration $\tau _{\text{pulse}}=1$ ps Eq. (%
\ref{s3a}) yields $W\gtrsim 20$ $\mu \text{J}$. Currently pico-second lasers
are commercially available with much greater energy.

\subsection{Population transfer from $3~^{3}\text{D}$ to $2~^{1}$P: level degeneracy problem}

Once the population is transferred to the $3~^{3}\text{D}$ state a strong
resonant driving field is applied on the $3~^{3}\text{D}$ to $2~^{1}\text{P}$
transition. However, levels $3~^{3}$D and $3~^{1}$D are essentially
degenerate (splitting between them is $0.2$ nm) and the applied external
field inevitably drives transition $3~^{1}\text{D}$ to $2~^{1}$P which is
more then 1000 times stronger then $3~^{3}$D to $2~^{1}$P transition (see
Fig. \ref{He}). Thus, the Rabi frequency which drives the $3~^{1}$D to $%
2~^{1}$P transition is much larger then those driving the $3~^{3}$D to $%
2~^{1}$P transition. Under such condition, practically no population can be
transferred from the $3~^{3}$D state as shown in Fig. \ref{f1}. This is true
no matter how strong the driving field is.

To avoid this problem it has been suggested to apply an additional resonant
field which drives the $3~^{1}\text{D}$ to $4~^{1}\text{P transition. The
  equivalent scheme is sketched in the insert of Fig. }$\ref{f2}. If the
additional driving field $\Omega _{bc}(t)$ is strong enough the population
can be transferred completely from $3~^{3}\text{D}$ to $2~^{1}\text{P (see
  Fig. }$\ref{f2}). The condition for an efficient population transfer is that
the amplitude $A$ of the additional field Rabi frequency $\Omega _{bc}(t)$
is greater then those of $\Omega _{cd}(t)$. This is demonstrated in Fig. \ref%
{f3} which shows population transfer as a function of $A$.

Result shown in the figures are obtained by numerical solution of the
evolution equations for $C_{a}$, $C_{b}$, $C_{c}$ and $C_{d}$ which are
probability amplitudes to find the system in the states $a$, $b$, $c$ and $d$
respectively. For resonant driving fields the evolution equations read%
\begin{equation}
  \frac{dC_{a}}{dt}=i\Omega _{ad}(t)C_{d},
\end{equation}%
\begin{equation}
  \frac{dC_{d}}{dt}=i\Omega _{ad}(t)C_{a}+i\Omega _{cd}(t)C_{c},
\end{equation}%
\begin{equation}
  \frac{dC_{c}}{dt}=i\Omega _{ad}(t)C_{d}+i\Omega _{bc}(t)C_{b},
\end{equation}%
\begin{equation}
  \frac{dC_{b}}{dt}=i\Omega _{bc}(t)C_{c},
\end{equation}%
with the initial condition $C_{a}(0)=1$, $C_{b}(0)=C_{c}(0)=C_{d}(0)=0$.
Rubi frequencies $\Omega (t)$ are given in the figures in dimensionless
units so that unit of time is the inverse amplitude of $\Omega _{ad}(t)$.

Derivation of these equations is provided in Appendix B for the special case
of a two level system.

\subsection{Discussion}

There are several methods for producing extreme ultra-violet lasing: for
example, using a capillary discharge \cite{Rocc94}, a free-electron laser
\cite{Milt01}, optical field ionization of a gas cell \cite{Lemo95} or
plasma-based recombination lasers \cite{Suck85}. Coherent XUV radiation can
also be produced by the generation of harmonics of an optical laser in a gas
or plasma medium. Our main goal is to investigate the extent to which
coherence effects might be useful in this problem.

Electron excitation has been the mechanism of choice for the pumping of a
wide variety of XUV\ lasers. Alternatively, high-intensity ultrashort (with
pulse duration less then $100$ fs) optical pulses can be used to pump
recombination lasers \cite{Burn89}. In this method, intense circularly
polarized light ionizes atoms via tunneling process. Then atoms recombine
yielding species in excited electron states.

The three-body recombination scheme is attractive due to its potential of
achieving lasing at XUV wavelengths with relatively moderate pumping
requirements. Several experiments have demonstrated gain and lasing in such
scheme \cite{Naga93,Krus96,Koro96}. Recombination mechanism relies on
obtaining ions in a relatively cold plasma which is possible due to short
duration of the pump pulse. Then rapid recombination and de-excitation
processes follow during which transient population inversion can be created.

In the present work we are focusing on lasing in He and He-like ions which
utilizes advantages of the recombination XUV lasers and possibly the effects
of quantum coherence. The later, for example, is the key for lasing without
inversion wherein quantum coherence created in the medium by means of strong
driving field helps to partially eliminate resonant absorption on the
transition of interest and to achieve gain without population inversion.
Such an effect holds promise for obtaining short wavelength lasers in the UV
and $X-$ray spectral domains where inverted medium is difficult to prepare
due to fast spontaneous decay.

\subsection{Condition of optimum STIRAP}

STIRAP is optimum if during the time of overlap of the two pulses ($\Omega
_{1}(t)$ and $\Omega _{2}(t)$) the effective Rabi frequency
\begin{equation}
  \Omega _{\text{eff}}(t)=\sqrt{\Omega _{1}^{2}(t)+\Omega _{2}^{2}(t)}
\end{equation}%
is constant \cite{Berg98}. This implies that population transfer is
efficient if the peak values of $\Omega _{1}$ and $\Omega _{2}$ are equal.

Because transition 2$^{3}$P$\rightarrow $2$^{3}$S is about $6$ times weaker
then transition 3$^{3}$D$\rightarrow $2$^{3}$P, the 2$^{3}$P$\rightarrow $2$%
^{3}$S transition determines the required energy per pulse. One can estimate
the energy from the condition%
\begin{equation}
  \Omega \tau =2\pi ,  \label{s0}
\end{equation}%
where $\tau $ is the pulse duration,
\begin{equation}
  \Omega =\frac{\wp E_{0}}{\hbar }  \label{s0a}
\end{equation}%
is the Rabi frequency for the 2$^{3}$P$\rightarrow $2$^{3}$S transition, $%
E_{0}$ is the amplitude of the electric field and $\wp $ is the
electric-dipole transition matrix element.

Condition (\ref{s0}) is essentially requirement of adiabaticity ($\Omega
\tau \gg 1$). Eqs. (\ref{s0}) and (\ref{s0a}) yield%
\begin{equation}
  E_{0}=\frac{2\pi \hbar }{\wp \tau }.
\end{equation}%
Energy density in electromagnetic wave is
\begin{equation}
  u=\frac{1}{2}\epsilon _{0}E_{0}^{2},
\end{equation}%
so that the total energy in pulse reads%
\begin{equation}
  W=uc\tau S=\frac{1}{2}\epsilon _{0}E_{0}^{2}c\tau S=2\pi ^{2}\epsilon _{0}cS%
  \frac{\hbar ^{2}}{\wp ^{2}\tau },  \label{s1}
\end{equation}%
where $S$ is the cross section area of the pulse. The 2$^{3}$P$\rightarrow $2%
$^{3}$S transition rate $\gamma $ can be estimated as%
\begin{equation}
  \gamma =\frac{k_{0}^{3}\wp ^{2}}{2\pi \epsilon _{0}\hbar }\quad \text{which
    yields }\quad \wp ^{2}=\frac{2\pi \epsilon _{0}\hbar \gamma }{k_{0}^{3}}.
  \label{s2}
\end{equation}%
Substitute Eq. (\ref{s2}) into Eq. (\ref{s1}) gives finally%
\begin{equation}
  W=8\pi ^{4}\frac{\hbar cS}{\gamma \tau \lambda ^{3}},  \label{s3}
\end{equation}%
where $\lambda $ is the wavelength of the transition.

\subsection{Derivation of evolution equation for a two-level system}

Here we consider a two level atom ($a$ is excited and $b$ is the ground
state). States $a$ and $b$ are described by wavefunctions $\psi _{a}$ and $%
\psi _{b}$. We assume that atom interacts with an applied electromagnetic
field which is polarized along the $x-$ axis%
\begin{equation}
  \vec{E}(t)=\hat{x}E_{0}\cos (\nu t).
\end{equation}%
Hamiltonian of the system reads
\begin{equation}
  \hat{H}=\hat{H}_{0}+\hat{V},
\end{equation}%
where $\hat{H}_{0}$ is the Hamiltonian of the free atom%
\begin{equation}
  \hat{H}_{0}\psi _{a}=\hbar \omega _{a}\psi _{a},\qquad \hat{H}_{0}\psi
  _{b}=\hbar \omega _{b}\psi _{b}
\end{equation}%
and $\hat{V}$ corresponds to the atom-field interaction in the dipole
approximation%
\begin{equation}
  \hat{V}=-e\vec{E}(t)\cdot \vec{r}=-exE_{0}\cos (\nu t),
\end{equation}%
$\vec{r}$ is the electron position in the atom and $e$ is the electron
charge.

Matrix elements are given by%
\begin{equation}
  <\psi _{a}|\hat{H}|\psi _{a}>=<\psi _{a}|\hat{H}_{0}|\psi _{a}>+<\psi _{a}|%
  \hat{V}|\psi _{a}>=\hbar \omega _{a},
\end{equation}%
\begin{equation}
  <\psi _{b}|\hat{H}|\psi _{b}>=<\psi _{b}|\hat{H}_{0}|\psi _{b}>+<\psi _{b}|%
  \hat{V}|\psi _{b}>=\hbar \omega _{b},
\end{equation}%
\begin{equation}
  <\psi _{a}|\hat{H}|\psi _{b}>=<\psi _{a}|\hat{H}_{0}|\psi _{b}>+<\psi _{a}|%
  \hat{V}|\psi _{b}>=<\psi _{a}|\hat{V}|\psi _{b}>=-eE_{0}\cos (\nu t)\int
  d^{3}r\psi _{a}^{\ast }x\psi _{b}.
\end{equation}%
We introduce the following notation for the dipole matrix element between
levels $a$ and $b$%
\begin{equation}
  \wp =e\int d^{3}r\psi _{a}^{\ast }x\psi _{b}
\end{equation}%
so that%
\begin{equation}
  <\psi _{a}|\hat{H}|\psi _{b}>=-\wp E_{0}\cos (\nu t).
\end{equation}

Under the applied electromagnetic field the atomic wavefunction $\psi $
evolves so that $\psi $ is a superposition of $\psi _{a}$ and $\psi _{b}$
\begin{equation}
  \psi =C_{a}(t)\psi _{a}+C_{b}(t)\psi _{b}  \label{a1}
\end{equation}%
with time-dependent coefficients $C_{a}(t)$ and $C_{b}(t)$ which we need to
find. $C_{a}(t)$ and $C_{b}(t)$ have meaning of probability amplitudes to
find atom in the state $a$ and $b$ respectively.

Schr\"{o}dinger equation for the atomic wavefunction $\psi $ reads%
\begin{equation}
  i\hbar \frac{\partial \psi }{\partial t}=\hat{H}\psi .  \label{a2}
\end{equation}

Substituting Eq. (\ref{a1}) into Eq. (\ref{a2}) we obtain%
\begin{equation}
  i\hbar \dot{C}_{a}(t)\psi _{a}+i\hbar \dot{C}_{b}(t)\psi _{b}=C_{a}(t)\hat{H}%
  \psi _{a}+C_{b}(t)\hat{H}\psi _{b}.  \label{a3}
\end{equation}%
Multiplying both sides of Eq. (\ref{a3}) by $\psi _{a}^{\ast }$ and
integration over the volume we find
\begin{equation}
  i\hbar \dot{C}_{a}(t)=C_{a}(t)<\psi _{a}|\hat{H}|\psi _{a}>+C_{b}(t)<\psi
  _{a}|\hat{H}|\psi _{b}>
\end{equation}%
or
\begin{equation}
  i\hbar \dot{C}_{a}(t)=\hbar \omega _{a}C_{a}(t)-\wp E_{0}\cos (\nu
  t)C_{b}(t).  \label{a4}
\end{equation}%
Multiplication of Eq. (\ref{a3}) by $\psi _{b}^{\ast }$ and integration over
volume yields%
\begin{equation}
  i\hbar \dot{C}_{b}(t)=\hbar \omega _{b}C_{b}(t)-\wp E_{0}\cos (\nu
  t)C_{a}(t).  \label{a5}
\end{equation}%
Dividing both sides by $i\hbar $ and incorporating decay rates $\gamma _{a}$
and $\gamma _{b}$ of the states $a$ and $b$ due to spontaneous emission and
collisions Eqs. (\ref{a4}) and (\ref{a5}) reduce to
\begin{equation}
  \dot{C}_{a}=-i\omega _{a}C_{a}+i\Omega \cos (\nu t)C_{b}-\frac{\gamma _{a}}{2%
  }C_{a},  \label{a6}
\end{equation}%
\begin{equation}
  \dot{C}_{b}=-i\omega _{b}C_{b}+i\Omega \cos (\nu t)C_{a}-\frac{\gamma _{b}}{2%
  }C_{b},  \label{a7}
\end{equation}%
where

\begin{equation*}
  \Omega =\frac{\wp E_{0}}{\hbar }\sim 10^{8}\div 10^{9}\text{ Hz.}
\end{equation*}%
Eqs. (\ref{a6}) and (\ref{a7}) determine $C_{a}(t)$ and $C_{b}(t)$ and,
thus, give a complete description of the system evolution. Introducing new
functions%
\begin{equation}
  C_{a}=\exp \left( -\left[ i\omega _{a}+\frac{\gamma _{a}}{2}\right] t\right)
  \tilde{C}_{a},
\end{equation}%
\begin{equation}
  C_{b}=\exp \left( -\left[ i\omega _{b}+\frac{\gamma _{b}}{2}\right] t\right)
  \tilde{C}_{b},
\end{equation}%
Eqs. (\ref{a6}) and (\ref{a7}) reduce to
\begin{equation}
  \frac{d\tilde{C}_{a}}{dt}=i\Omega \cos (\nu t)\exp \left( \left[ i\omega +%
      \frac{\gamma _{a}-\gamma _{b}}{2}\right] t\right) \tilde{C}_{b},  \label{a8}
\end{equation}%
\begin{equation}
  \frac{d\tilde{C}_{b}}{dt}=i\Omega \cos (\nu t)\exp \left( -\left[ i\omega +%
      \frac{\gamma _{a}-\gamma _{b}}{2}\right] t\right) \tilde{C}_{a},  \label{a9}
\end{equation}%
where%
\begin{equation*}
  \omega =\omega _{a}-\omega _{b}.
\end{equation*}%
Taking into account that
\begin{equation*}
  \cos (\nu t)=\frac{1}{2}\left( e^{i\nu t}+e^{-i\nu t}\right)
\end{equation*}%
we find%
\begin{equation}
  \frac{d\tilde{C}_{a}}{dt}=i\frac{\Omega }{2}\left[ \exp \left( \left[
        i(\omega +\nu )+\frac{\gamma _{a}-\gamma _{b}}{2}\right] t\right) +\exp
    \left( \left[ i(\omega -\nu )+\frac{\gamma _{a}-\gamma _{b}}{2}\right]
      t\right) \right] \tilde{C}_{b},  \label{a10}
\end{equation}%
\begin{equation}
  \frac{d\tilde{C}_{b}}{dt}=i\frac{\Omega }{2}\left[ \exp \left( -\left[
        i(\omega +\nu )+\frac{\gamma _{a}-\gamma _{b}}{2}\right] t\right) +\exp
    \left( -\left[ i(\omega -\nu )+\frac{\gamma _{a}-\gamma _{b}}{2}\right]
      t\right) \right] \tilde{C}_{a}.  \label{a11}
\end{equation}%
Next we omit the fast oscillating terms (containing $\omega +\nu $). Such
simplification is known as the Rotating Wave Approximation (RWA). In the RWA
Eqs. (\ref{a10}) and (\ref{a11}) reduce to
\begin{equation}
  \frac{d\tilde{C}_{a}}{dt}=i\frac{\Omega }{2}\exp \left( i\Delta t\right)
  \tilde{C}_{b},  \label{a12}
\end{equation}%
\begin{equation}
  \frac{d\tilde{C}_{b}}{dt}=i\frac{\Omega }{2}\exp \left( -i\Delta t\right)
  \tilde{C}_{a},  \label{a13}
\end{equation}%
where%
\begin{equation*}
  \Delta =\omega -\nu -i\frac{(\gamma _{a}-\gamma _{b})}{2}
\end{equation*}%
is the detuning of the frequency of the applied field $\nu $ from the atomic
transition frequency $\omega =\omega _{a}-\omega _{b}$.

\section{LWI}
\subsection{Quantum Coherence}
\subsection{LWI}
\subsection{Superradiance}
\section{Motivation}
Go to higher frequency harder.
Transient regime can't have steady state.

%\input{ch-intro/intro_contributions}
