\chapter{Experiment\label{ch:experiment}}
\section{Apparatus}
Lasing system with a Ti:Sapphire amplifier lasing @ $800nm$, with a
pulse duration of $50ns$ and output energy of $3mJ$.

Optical parametric amplifier with a tunable wavelength wavelength
between $500nm ~ 2\mu m$, max energy is $100mJ$.
\section{Population probing techniques}
\subsection{Emission}
The first thing we want to study is probing the population at
different stages of ionized Helium. When you ionize Helium, the
electrons will recombine, cascaded down through collision to lower
state. So we want to know populations of certain level at certain
stage.

Several techniques have been developed to observe the behavior of this
ionized Helium.

The intense ionizing beam which is $800nm$ is focused to generate
ionized Helium and a plasma.
picture.

One of the emission lines of Helium which is dominating in visible, is
$587nm$.

Quantum weighs of triplet are 3 times bigger than singlet, so even the
A and B coefficient are similar, we see stronger lines.

Pressure broadening:
This is emission line at different pressure. The higher pressure you
have , the broader line width you will get.

Gated CCD spectra:
you record the spectra at different time of the Helium ionization, so
you get the dynamics of the emission of 587 lines, as a function of
time. From the spectrum, we could see that the peak occurs roughly at
$10ns$ after ionization. Besides, the width of the spectra is
shrinking as time goes by. The reason is that, after the ionization,
the density of the electron is high, and the width is due to collision
between electrons and atoms. As the plasma expand, you can see the
line gets narrower.

At different delay time, you can see different width of the spectra line.
\subsection{Transmission study}
Another technique is also developed to observe the population at
certain levels, by sending in a femto-second laser pulse, resonating
with one of the transitions.

One of the advantage of using femto-second laser is that it has very
wide spectrum, much wider than the transition itself.

So you could see absorption directly at the transmission spectra, of
the transmitted probe beam.

\subsection{Absorption vs Delay}
Typical spectra of femto-second pulse without ionizing Helium are 50fs
wide. As the time progress, you can see bigger and bigger effects of
the absorptio due to the accumulation of population of $2^3S$ over
time. This is done at fixed pressure.

\subsection{Absorption vs probe energy}
At different probing energy, you can see saturation effects because
depletion of population.

\subsection{linewidth as a function of pressure}
Another transition at $587nm$, which you also see this absorption
effects, we can carefully measure the line width at different
pressure.

Low pressure, it's linear, at higher pressure, because of saturation,
it's non-linear.

We have developed a method for probing the population evolution of
excited atoms, we are going to use this not only to measure the
population, but also linewidth, such as collision rates.

\section{Broadening Analysis}
Experiments show the line broadening of $587nm$ ($2^3P->3^3D$) is
$0.5nm$, while $1080nm$ ($2^3s->2^3P$), it's $0.4nm$

The Experiment Parameters are:
Helium: $200mbar$, the atom density is $5x10^{18}cm^{-3}$
Probe beam delay: $10ns$
Electron density: $2x10^{17}cm^{-3}$
Atom density(excited): $10^{14}cm^{-3}$
Electron temperature: $0.5eV$

For a rough estimation of electron-atom collision cross section, we
could use the equation as follows:
\begin{equation}
\sigma_{ea} = (n^2a_B)^2; \tau_{ea}=\frac{1}{N_e v_e \sigma_{ea}} =
212 ps (for 1.08 \mu m)
\end{equation}
picture.

There are different broadening mechanism:
- Natural: quite narrow, usually on the order of $1/1000$ $\Delta
\lambda_{collision}$, so could be neglected.
- Doppler: Due to thermal motion of the emitter.
- Instrumental: Equals to slit width times dispersion by grating.
- Collisional: When emitting, if the emitter colloided by other
particles, the emitting process will be disturbed.
- Other: Like the superradiance?

\subsection{Doppler Broadening}
\begin{equation}
\Delta w = 2 \sqrt{ln2} \sqrt{ \frac{2kT \lambda^2_0}{mc^2} }
\end{equation}

For $587nm$: $\Delta w_D = 0.32 nm$ while total is $5 \AA$.
For $1080nm$: $\Delta w_D = 0.59 nm$ while total is $24 \AA$.

So the Doppler broadening could be neglected.

\subsection{Instrumental Broadening}
Slit width: $40\mu m$

Instrumental broadening: $4 \AA$

Subtract instrumental from total:

587: $\sqrt{0.5^2-0.4^2} = 0.3 nm ~ 2.1 ps$

1080: $\sqrt{2.4^2-0.4^2} = 2.36 nm ~ 0.6 ps$

\subsection{Collisional Broadening}
Picture.

There's atom-atom collision and atom-election collision
(?electron-electron?)

But since the electron density ($10^{17}cm^{-3}$) is much higher than
atom density ($10^{14}cm^{-3}$), so the electron-atom collision will
dominate.

The weight of atoms are much higher than electrons, so atoms could not
gain much speed when laser present, while electron is fast!

Electron atom collision includes electron-impact excitation and
ionization, and their reverse processes, so one only need to consider
one of those processes.

Our electron is cold, only $0.5eV$, so the ionization cross section is
small, which could be neglected.

After the laser passed, it could be assumed that there's no photon, so
all photon related process like photo ionization/excitation could be
neglected.

Our electron density is high $10^{17}cm^{-3}$, so after laser passed,
electron reaches their local thermal equilibrium (LTE)(ref) quickly, at
$10ns$. Of course, Maxwellian distribution could be assumed.

To summarize, we just need to consider electron impact excitation,
under electron Maxwellian distribution.

From the Ralchenko paper(ref), we found the cross section fitting
curve, for electron impact transition between different states. This
curve is trustable, for it match experiments quite well.

The cross section is given by:
\begin{equation}
\sigma(E,\Delta E) = a_0^2 \pi \frac{R_y}{g_l E}
\Omega(\frac{E}{\Delta E})
\end{equation}
Where $a_0$ is just Bohr radius, $\pi a_0^2 = 0.8797 * 10^{-16} cm^2$,
$g_l$ is the statistical weight, for $2^3 S$ and $2^3 P$, $g_l = 3$.
$R_y = 13.6057 eV$ is the Rydberg energy. And $E$ is the collisional
energy, where $E=\frac{1}{2}mv^2$, $v$ could be calculated by
averaging through a Maxwellian distribution.

picture.

$\Delta E$ is the energy difference between initial and final states,
for $587nm$: it's $\frac{1.24}{0.587} eV$, while for $1080nm$, it's
$\frac{1.24}{1.08} eV$.

\begin{equation}
\Omega(x) = (A_1 ln(x) + A_2 + \frac{A_3}{x} + \frac{A_4}{x^2} +
\frac{A_5}{x^3}) (\frac{x+1}{x+A_6})
\end{equation}

For $2^3S -> 2^3P$, those coefficients are:

\begin{tabular}{ l | r }
  $A_1$ & $7.696 * 10^1$ \\
  $A_2$ & $1.250 * 10^2$ \\
  $A_3$ & $4.938 * 10^1$ \\
  $A_4$ & $-4.778 * 10^1$ \\
  $A_5$ & $3.189 * 10^2$ \\
  $A_6$ & $8.157$
\end{tabular}

While for $2^3S -> 2^3P$, those coefficients are:

\begin{tabular}{ l | r }
  $A_1$ & $1.414 * 10^2$ \\
  $A_2$ & $9.031 * 10^1$ \\
  $A_3$ & $-6.238 * 10^2$ \\
  $A_4$ & $1.183 * 10^3$ \\
  $A_5$ & $-6.424 * 10^2$ \\
  $A_6$ & $8.626$
\end{tabular}

So now we got everything to estimate $\sigma v$
\begin{equation}
\sigma(E,\Delta E) = \pi a_0^2 \frac{R_y}{g_l E}\Omega(\frac{E}{\Delta
E})
\end{equation}

Since we have Maxwellian distribution:
\begin{equation}
f(v) dv = (\frac{m}{2\pi k T})^(\frac{3}{2})4\pi v^2
exp(\frac{-mv^2}{2kT}) dv
\end{equation}

by switching to energy using, $E=\frac{1}{2}mv^2$, $v=\sqrt{2E}{m}, dv
=\frac{1}{\sqrt{2mE}}dE$, we got:
\begin{equation}
f(E)dE=(\frac{m}{2\pi k T})^{\frac{3}{2}}4\pi \frac{2E}{m} exp(-
\frac{E}{kT}) \frac{1}{\sqrt{2mE}} dE
\end{equation}
So we get:
\begin{equation}
<\sigma v> = \int_0^{+\inf} \sigma(E)\sqrt{\frac{2E}{m}}f(E)dE
\end{equation}
Then using,
\begin{equation}
\tau_{ea} = \frac{1}{N_e<\sigma v>}
\end{equation}
Where $N_e=2*10^{17}cm^{-3}$

By using Mathematica to compute the integral, it turns out that
$\tau_{ea}$ is quite small. This results is reasonable, for our
electron is cold, $kT=0.5eV$, this is small compared to $3^3D$ ->
$2^3P$ and $2^3P$ -> $2^3S$ gap, so the cross section should be small.

Picture.

However this is not the end of the story, this small result enlightens
us to focus on the electron-impact transition, whose cross section is
big in our case. Apparently, the smaller the energy difference,
between the initial and final state, the easier to excite, so the
cross section is bigger.

The total cross section is the sum of all the transitions, so let's
focus on the dominating one first.

For final state $3^3D$, the most close energy level is $3^1D$, and for
$2^3P$, it's $2^1P$. Now we look for the cross section of these two
transitions.

From graph 15 on page 617, $2^3P$ -> $2^1P$, cross section reads
$\sigma = 10^{-15}cm^2$ @ $0.5eV$

around $0.5eV$, the curve is flat enough to avoid taking average from
Maxwellian distribution.
\begin{equation*}
<v> = 2*10^7 cm/s
\end{equation*}
\begin{equation*}
<\sigma v> = \sigma <v>
\end{equation*}
\begin{equation}
\tau_{ea} = \frac{1}{N_e<\sigma v>} = \frac{1}{N_e \sigma <v>}
=\frac{1}{2*10^{17}cm^{-3}*10^{-15}cm^2*2*10^7cm/s} = 250 ps
\end{equation}

Similarly, by graph 28 on page 619, we found excitation from $3^3D$ ->
$3^1D$ dominates, whose cross section reads:
\begin{equation}
\sigma = 4.5 * 10^{-15} cm^2
\tau_{ea} = \frac{1}{N_e\sigma<v>} = 55ps
\end{equation}

But this could not explain all the broadening from experiments:

\begin{tabular}{ l c r }
       & Experiment & Calculation \\
  587  & 0.3nm(2.1ps) & 55ps \\
  1080 & 2.2nm(0.6ps) & 250ps \\
\end{tabular}

This indicates, there should be broadening factors brought in by other
effects. It should be superradiance. In that case, only the cylinder
of atoms contributes to superradiance, whose radius is $\lambda$, and
the length is plasma length.

Picture.

For 587nm, the single atom decay is given by:
\begin{equation}
\tau = \frac{1}{A} = 14ns
\end{equation}
The superradiance factor is:$N_a\lambda^2L/2\pi$, where
$N_a=10^{14}cm^{-3}$, $\lambda=587nm$, $L=0.7cm$, so the broadening
brought in by superradiance is:
\begin{equation}
\tau_s=\frac{\tau}{N_a\lambda^2L/2\pi} \\
= \frac{14ns}{10^{14}*0.587^2*10^{-8}*0.7/2\pi} = 0.364ps
\end{equation}
compared with $0.3ps$ experimental measurements, they are on the same
order.

For $1080nm$, $N_a$ is larger, and
\begin{equation}
\tau=\frac{1}{A}=100ns
\end{equation}
\begin{equation}
\tau_s=\frac{\tau}{N_a\lambda^2L/2\pi}=\frac{100ns}{4*10^{14}*1.08^2*0.7*10^{-8}/2\pi}=0.192ps
\end{equation}
While the width measured from experiment is $0.2ps$, they are on the
same order.

\section{Simulation Results}
If after ionization, a Maxwellian distribution could be assumed, it
will be quite handy. For only two parameters are needed to describe
the whole system. Otherwise, a more complicated method need to be take
to handle the distribution, like the code we are using, which traces
all the particles one by one.


\section{Superradiance}
\section{Carbon line identification}
%\section{Preliminary data and results on emission and absorption by excited Helium atoms}


%\section{Options}
\label{sec:usage:options}

In this section, we describe the options you can set when using this thesis class.
\tablespacing
% tablespacing is defined by the class to set single spacing for the long table when in doublespacing mode. If the singlespace option is set, this command has no effect.

\begin{longtable}{p{0.3\linewidth} p{0.6\linewidth}}

  % First page heading
  \caption[Options Provided by the PUthesis Class]{List of options for the puthesis document class and template} \label{tab:usage:options}\\
  \toprule
  \textbf{Option} & \textbf{Description} \\
  \midrule
  \endfirsthead

  % Future page heading
  \caption[]{(continued)}\\
  \toprule
  \textbf{Option} & \textbf{Description} \\
  \midrule
  \endhead

  % Page footer
  \midrule
  \multicolumn{2}{r}{(Continued on next page)}\\
  \endfoot

  % Last page footer
  \bottomrule
  \endlastfoot

  12pt &
  Specify the font size for body text as a parameter to \texttt{documentclass}. The Mudd Library requirements state that 12pt is preferred for serif fonts (e.g., Times New Roman) and 10pt for sans-serif fonts (e.g., Arial).
  \\

  letterpaper &
  If your document is coming out in a4paper, your LaTeX defaults may be wrong. Set this option as a parameter to \texttt{documentclass} to have the correct 8.5"x11" paper size.
  \\

  lot &
  Set this option as a parameter to \texttt{documentclass} to insert a List of Tables after the Table of Contents.
  \\


  lof &
  Set this option as a parameter to \texttt{documentclass} to insert a List of Figures after the Table of Contents and the List of Figures.
  \\

  los &
  Set this option as a parameter to \texttt{documentclass} to insert a List of Symbols after the Table of Contents and the other lists.
  \\

  singlespace &
  Set this option as a parameter to \texttt{documentclass} to single space your document. Double spacing is the default otherwise, and is required for the electronic copy you submit to ProQuest. Single spacing is permitted for the printed and bound copies for Mudd Library.
  \\

  draft &
  Set this option as a parameter to \texttt{documentclass} to have \LaTeX mark sections of your document that have formatting errors (e.g., overfull hboxes).
  \\

  % the cmidrule here spans both columns but is indented slightly on the left and right.
  \cmidrule[0.1pt](l{0.5em}r{0.5em}){1-2}

  \raggedright
  $\backslash newcommand$ $\{\backslash printmode\}\{\}$ &
  Insert this command after the \texttt{documentclass} command to turn off the hyperref package to produce a PDF suitable for printing.
  \\

  \raggedright
  $\backslash newcommand$ $\{\backslash proquestmode\}\{\}$  &
  Insert this command after the \texttt{documentclass} command to turn off the `colorlinks' option to the hyperref package. Links in the pdf document will then be outlined in color instead of having the text itself be colored. This is more suitable when the PDF may be viewed online or printed by the reader.
  \\

  $\backslash makefrontmatter$ &
  Insert this command after the \texttt{$\backslash begin\{document\}$} command, but before including your chapters to insert the Table of Contents and other front matter.
  \\

  \cmidrule[0.1pt](l{0.5em}r{0.5em}){1-2}

  $\backslash title$ &
  Set the title of your dissertation. Used on the title page and in the PDF properties.
  \\

  $\backslash submitted$ &
  Set the submission date of your dissertation. Used on the title page. This should be the month and year when your degree will be conferred, generally only January, April, June, September, or November. Check the Mudd Library rules~\cite{mudd2009} for the appropriate deadlines.
  \\

  $\backslash copyrightyear$ &
  Set the submission year of your dissertation. Used on the copyright page.
  \\

  $\backslash author$ &
  Your full name. Used on the title page, copyright page, and the PDF properties. \\

  $\backslash adviser$ &
  Your adviser's full name. Used on the title page. \\

  $\backslash departmentprefix$ &
  The wording that precedes your department or program name. Used on the title page. The default is ``Department of'', since most people list their department and can leave this out (e.g., Department of Electrical Engineering), however if yours is a program, set $\backslash departmentprefix\{Program in\}$ \\

  $\backslash department$ &
  The name of your department or program. Used on the title page. \\

  \cmidrule[0.1pt](l{0.5em}r{0.5em}){1-2}

  \raggedright
  $\backslash renewcommand$ $\{\backslash maketitlepage\}\{\}$ &
  Disable the insertion of the title page in the front matter. This is useful for early drafts of your dissertation. \\

  \raggedright  % full justification places the * in an awkward place
  $\backslash renewcommand*\{\backslash makecopyrightpage\}\{\}$ &
  Disable the insertion of the copyright page in the front matter. This is useful for early drafts of your dissertation. \\

  \raggedright
  $\backslash renewcommand*\{\backslash makeabstract\}\{\}$ &
  Disable the insertion of the abstract in the front matter. This is useful for early drafts of your dissertation. \\

\end{longtable}
\bodyspacing
% bodyspacing restores double spacing or single spacing after the table

% need blank space after \bodyspacing

I've seen other people print their dissertations using $\backslash pagestyle\{headings\}$, which places running headings on the top of each page with the chapter number, chapter name, and page number. This documentclass is not currently compatible with this option -- the margins are setup to be correct with page numbers in the footer, placing them 3/4" from the edge of the paper, as required. If you wish to use headings, you will need to adjust the margins accordingly.




