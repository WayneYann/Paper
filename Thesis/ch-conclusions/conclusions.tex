\chapter{Conclusions}\label{ch:conclusions}

This disseration represents several advancements in the understanding of the chemistry-transport coupling in flame dynamics and emissions.  In particular, two core topics are mainly focused on: flame stabilization and soot evolution in reacting flows at engine relevant conditions.  To unravel the coupling effects of chemical kinetics and flow dynamics, for both topics, foundational studies on chemical kinetics are first conducted in relatively simple, one-dimensional, laminar counterflow or stagnation-flow configurations, facilitated with the combination of experiments and computations.  Complexities from the flow conditions are gradually brought in to elucidate the interactions between chemistry and transport at elevated temperatures and pressures or turbulent conditions.  Major contributions from this dissertation are summarized below.

\section{Major Contributions}

\subsection{Characterization of Nonpremixed Cool Flames}

In Chapter~\ref{ch:NTC}, experimental substantiation was acquired for the computationally predicted existence of low-temperature, NTC-affected, weakly burning nonpremixed flames in the counterflow configuration.  In particular, the filtered PMT imaging demonstrated the presence of the signature CH$_2$O chemiluminescence in the counterflowing heated air stream against nitrogen-diluted DME, while sensitive infrared imaging determined the corresponding ignition temperature.  Extensive experimentation then demonstrated that the low-temperature reactivity was enhanced with increasing air temperature, decreasing strain rate of the flow, and was insensitive to the DME concentration in the fuel stream.  Accomanying computations substantiated the experimental results, and further corroborated the essential NTC chemistry governing the observed phenomena.

Furthermore, for the first time, the hysteretic ignition and extinction behavior of the nonpremixed cool flame was experimentally observed and quantified at elevated pressures.  Results further show that, although low-temperature chemistry is crucial for the initiation and sustain of the cool flame, the dominant chemical pathways shift from reactions responsible for low-temperature radical runaway to cool flame heat release reactions upon ignition.  The heat release from the cool flame is able to sustain itself at lower oxidizer boundary temperature and, therefore, results in the hysteresis temperature window between ignition and extinction.

Increasing ambient pressure and/or oxygen concentration in the oxidizer stream promotes the heat release from low-temperature chemistry and extends the hysteresis between ignition and extinction.  Although the influences on the cool flame ignition temperature were well predicted by computation, the influences on extinction were significantly overpredicted.  Possible reasons for such discrepancies were discussed, including the uncertainties from experiments and chemical models. The need for improved comprehensiveness of the chemical kinetics model is emphasized. 

\subsection{Demonstration of Autoignition-Affected Flame Dynamics}

In Chapter~\ref{ch:dynamics}, computations of two-dimensional nonpremixed DME flames in heated air coflows were presented.  The computations were conducted at $30$ atmospheres to observe the influence of NTC chemistry on the stabilization mechanism.  The stabilization of steady flames was first studied, and the effects of boundary temperature and velocity were investigated.  

The heat release rate profile and characteristic species profiles for low- and high-temperature chemistry, autoignition, and partially premixed flame propagation were examined.  Further investigation based on Chemical Explosive Mode Analysis and Lagrangian Flamelet Analysis enabled the determination of the evolution of the controlling chemical pathways and the stabilization mechanism.  

For fixed fuel and coflow velocities of $3.2$ m/s and varying boundary temperatures, the $700$ K case was characterized as \emph {kinetically} stabilized, for, neglecting the diffusion processes along mixture fraction iso-contours, the one-dimensional LFA agrees with the two-dimensional CFD responses.  As the boundary temperature increases, the leading point of the heat release profile shifts to richer mixture fractions and then shifts back due to the NTC effect on the autoignition process and the coupling between autoignition and premixed flame propagation chemistry.  Stabilization is also affected by both inhomogeneous autoignition and premixed flame propagation, as in the $800$ and $900$ K cases.  The $1100$ K case was characterized as \emph {kinematically} stabilized, for it exhibits the classical triple flame structure, with stabilization achieved due to the balance between the premixed flame propagation velocity and the local incoming flow velocity.

For fixed boundary temperature of $900$ K and varying fuel and coflow velocities, at $2.4$ m/s, the lifted flame appears to be the classical triple flame stabilized by the balance between the local flame speed and incoming flow velocity, and it is therefore characterized as \emph {kinematically} stabilized.  As the inlet velocity increases, such a balance cannot be achieved at certain mixture fractions.  Instead, inhomogeneous autoignition becomes the dominant combustion mode.  As a consequence, the multibrachial structure is stabilized by both premixed flame propagation and inhomogeneous autoignition and is characterized as multimode stabilized.  At $8.0$ m/s, the \emph{kinematic} balance cannot be achieved anywhere in the flow field due to lack of back diffusion.  A \emph{kinetically} stabilized inhomogeneous autoignition front is formed where diffusion processes along the mixture fraction iso-contours are negligible compared to the gradient direction.  In the \emph{kinetically} stabilized autoignition front, NTC chemistry plays an important role, dictating the stabilization point.  This point occurs at the mixture fraction with the shortest ignition delay time, which is a compromise of the first stage autoignition delay time and heat release from low temperature chemistry.  Conversely, the \emph{kinematically} stabilized flames are less affected by NTC chemistry, with only a minor effect on the local flame speed resulting from the upstream accumulation of heat and radicals.  

An extended two-dimensional stabilization regime diagram was constructed, considering both transport (inlet velocity) and chemical (coflow boundary temperature) effects.  The stabilization regime diagram includes frozen flow, kinetically stabilized, autoignition-propagation-coupled stabilized, kinematically stabilized, and burner stabilized regimes.  At high coflow boundary temperatures or low inlet velocities, the classical tribrachial flame structure is achieved, and autoignition contributes less to the stabilization due to reduced heat and radical accumulation.  The kinematic balance between the local flow speed and flame propagation speed is the dominant stabilization mechanism.  On the contrary, kinetic stabilization is achieved at lower coflow temperatures or higher inlet velocities as autoignition becomes dominant.  Due to the transition of the dominant chemical pathways during autoignition, the kinetically stabilized structure is usually multibrachial.

The unsteady effects on flame dynamics in oscillating flows were then investigated with the comparisons with the steady cases.  The inlet velocity oscillates between $2.4$ and $8.0$ m/s at $25$, $50$, and $100$ Hz.  Flame dynamics in such oscillating flows and frequency effects on the hydrodynamics-chemistry coupling were analyzed.  

The heat release rate profiles were examined to describe the thermal structure.  The morphology of the thermal structure transitions between tribrachial and multibrachial.  The multibrachial structure is favored when the inlet velocity is higher, although there is hysteresis during the transition.  Such structures agree well with the steady cases, which correspond to different combustion modes: tribrachial flame and autoignition.  Normalized displacement velocity was defined to differentiate these two modes in the current study and compare with the steady cases.  

According to the steady results, the normalized displacement velocity for a tribrachial flame is around unity and is larger for autoignition.  The same criterion was applied to the unsteady cases to elucidate the evolution of combustion mode.  As the inlet velocity decreases, autoignition is the dominant combustion process until flame chemistry takes over around the most upstream location and slowest inlet velocity.  The tribrachial flame is convected downstream as the flow velocity increases.  The radical and heat accumulation upstream of the tribrachial flame finally results in autoignition, showing a sudden increase in the normalized displacement velocity. 

Oscillation frequency effects on the hydrodynamics-chemistry coupling were analyzed by examining the profiles of temperature, methyoxymethylperoxy radical, and hydrogen peroxide during the oscillation process.  It is found that, at the three frequencies investigated, the tribrachial structure does not have sufficient time to reach steady state, and the transition from tribrachial flame to a multibrachial autoignition front occurs over a finite induction time as velocity increases.  Consequently, the decreasing-velocity and increasing-velocity cycles have different normalized displacement velocities and hence demonstrate hysteresis.  At lower frequencies, such hysteresis is less pronounced, for longer relaxation time is allowed to approach the quasi-steady state condition.

NTC chemistry represented by methyoxymethylperoxy radical accumulation and depletion has shorter time scale and therefore is able to respond to the hydrodynamic changes.  However, autoignition and flame establishment have comparable time scales to the oscillation period and are therefore coupled with the flow dynamics.  At lower oscillation frequencies, chemical kinetics is closer to reaching the two steady-state conditions with the maximum and minimum boundary velocities.

\subsection{Characterization of Nonpremixed Sooting Limits}

In Chapter~\ref{ch:biofuel}, $n$-heptane, $n$-butanol, and methyl butanoate were chosen as diesel, bioalcohol, and biodiesel surrogates of interest due to their similar volatilities and consequent potential applications in diesel fuel blending. Their sooting limits were measured experimentally with the stagnation-flow apparatus. Computations were conducted for the same configuration using detailed PAH chemistry and a detailed soot model based on the Hybrid Method of Moments (HMOM). Argon dilution was adopted to keep the thermal environment for the three fuel cases essentially the same in order to elucidate the chemical effects.  Both experimental and computational results show the critical strain rates of the three fuels, based on the absolute soot volume fraction, increase with increasing oxygen mole fraction in the oxidizer due to the thermal effect. Moreover, although $n$-heptane and $n$-butanol show similar sooting propensities, methyl butanoate produces significantly less soot.

Sensitivity and reaction path analysis was performed for naphthalene, the first PAH species from which soot forms. This analysis revealed that, despite different sooting tendencies, the three fuels shared similar PAH chemical pathways. C$_5$, C$_6$, C$_7$ and C$_9$ rings as well as naphthalene are formed sequentially through the combination of C$_3$ and smaller chain radicals resulting from fuel cracking processes. Due to the fuel bound oxygen in methyl butanoate, less and shorter chain radicals are available for soot formation, compared with the other fuels, such that methyl butanoate forms the least soot and has the lowest critical strain rates.

The strain rate effects on soot formation were also examined. Despite different sooting propensities, for all three fuels, C$_5$ and C$_6$ ring formation reactions are the rate-limiting steps. Their concentrations drop as the residence time is reduced, such that the downstream PAH chemistry is consequently inhibited, resulting in the sooting limits.

\subsection{Demonstration of Hydrogen-Affected Soot Emissions}

In Chapter~\ref{ch:bluff}, a sooting turbulent bluff body stabilized ethylene/hydrogen flame was studied both experimentally and computationally and compared with a previously analyzed neat ethylene counterpart~\cite{mueller13}.  Laser-induced incandescence (LII) measurement of soot volume fraction was conducted at the University of Adelaide.  An integrated Large Eddy Simulation (LES) model was adopted to elucidate the interactions between soot, turbulence, and chemistry.  The statistical soot model was based on the Hybrid Method of Moments (HMOM), considering detailed nucleation, condensation, particle collision, surface growth, oxidation, and fragmentation processes that influence soot evolution.  The combustion model for the gas-phase was based on the Radiation Flamelet/Progress Variable (RFPV) model with modifications to account for the removal of Polycyclic Aromatic Hydrocarbon (PAH) from the gas-phase.  A presumed PDF was utilized to model unresolved subfilter scale interactions.

Similar to the neat ethylene bluff body flame, three distinct flow regions were observed experimentally: a sooting recirculation zone, a non-sooting, high-strain neck region, and a sooting jet-like zone.  Although the ethylene/hydrogen case is significantly less sooting than the neat ethylene counterpart overall, soot reduction in the recirculation zone near the bluff body is more pronounced than in the downstream jet-like region.  Both chemical and hydrodynamic effects were identified as reasons for this decrease.

The chemical effect was first quantified from a steady flamelet calculation.  Due to the hydrogen addition, PAH mass fraction was found to decrease by a factor of two, indicating a factor of four decrease in the PAH-based growth rate that includes both nucleation and PAH condensation.  This is consistent with the soot reduction in the downstream jet-like region, where PAH-based growth is dominant.  Therefore, the chemical effect is dominant in this region.

The hydrodynamic effect was then analyzed, since both the experimental measurements and LES calculations demonstrated additional soot reduction in the recirculation zone, which could not be explained with the chemical effect alone.  The streamline patterns in the recirculation zone of the two flames are similar according to LES, and therefore, differences in residence time plays little role in the difference in soot volume fraction in the recirculation zone.  

Conversely, in the experiment, to attain the same heat release rate and similar Reynolds number in the hydrogen added flame as the neat ethylene flame, the central jet velocity was increased.  This increase in the fuel to air coflow momentum flux ratio entrains less fuel into the recirculation zone near the bluff body surface, compared to the neat ethylene flame.  Due to the leaner mixture fraction in the recirculation zone, soot nucleation and surface growth is inhibited and soot oxidation is promoted.  This hydrodynamic effect together with the chemical effect accounts for the soot reduction in the recirculation zone.    
 
\section{Recommendations for Future Work}

While this dissertation has advanced the understanding of the coupling effects of chemistry and transport in flame dynamics and soot emissions, the understanding itself has in turn generated a plethora of interesting and relevant questions, the resolution of which would further extend the frontier of combuston science.  With due modesty, the author of this dissertation presents in the following some of these challenging questions in the hope of stimulating further research leading to closure.

\begin{itemize}
\item \textbf{Foundational Studies on Cool Flames:} As demonstrated in Chapter~\ref{ch:NTC}, the nonpremixed cool flame has distinctive states of ignition and extinction.  In a previous study by Zhao \emph{et al.}~\cite{zhao16}, the laminar propagation speeds of premixed cool flames were measured in a counterflow configuration.  It is quite clear that, just like the traditional hot flame, the cool flame is a new type of flame that can also propagate and has limiting phenomena.  Consequently, other foundational studies on cool flames, such as flammability limits and flame instabilities, should be pursued.  Such studies will not be trivial, mainly due to two reasons: limited experimental data and less accurate chemical models.  As demonstrated in Chapter~\ref{ch:NTC}, the experimental detection of cool flames is already very challenging, which also limits the development of low-temperature chemical models.  Much effort is needed in the quantification of low-temperature species in cool flames and providing validation targets for chemical model development.

\item \textbf{Flame Dynamics in Turbulent Flows:} Although Chapter~\ref{ch:dynamics} represents a useful contribution to the understanding of nonpremixed flames at autoignitive conditions, practical engines conditions are turbulent.  In such flows, the unsteady motion of eddies and enhanced temperature and species dissipation can strongly influence stabilization.  Therefore, further investigation is required to understand the role of turbulence in the stabilization of such flames and determine whether the stabilization mechanism is modified.  In terms of computational study, direct numerical simulation (DNS) for turbulent reacting flows at elevated pressures will be computationally expensive.  Therefore, large eddy simulation (LES) is a reasonable choice.  The combustion model in LES will then need to be able to capture the multi-modal combustion characteristics, as demonstrated in Chapter~\ref{ch:dynamics}.  In terms of experimental study, high pressure experiments are needed to provide potential experimental substantiation of these predicted multibrachial structures in laminar flows.  However, maintaining laminar air coflows at elevated temperatures and pressures and with large flow rates is also very challenging.  Moving on to turbulent studies, experimental techniques that can characterize different combustion modes need to be developed, such as iso-surface-tracking, which can be utilized to quantify the propagation speed of the reacting front and compared to the displacement velocity in computations.  To capture the low-temperature reactivity, online characterization of the low-temperature species such as formaldehyde would be helpful, however, it will be challenging to differentiate between the low-temperature and high-temperature chemistry, for formaldehyde is also an important intermediate species during high-temperature oxidation processes.

\item \textbf{Validation and Development of Soot Models:} As the regulation on soot emission has become very stringent, accurate soot prediction for practical combustion devices is needed.  To achieve this goal, two main issues need to be solved.  First, soot and gas-phase measurements are needed for chemical and combustion model development.  Since soot particles have strong black body effects, laser measurements of gaseous species are typically not available in high-loading sooting flames.  Especially in turbulent sooting flames, simultaneous measurements of soot, velocity, and other scalars are very rare.  Second, the development of soot models should be further advanced.  As discussed in Chapter~\ref{ch:biofuel}, well controlled laminar experiments can provide useful validation targets for soot model development.  Having sufficient confidence in the PAH and soot models, turbulent flames with low soot loading could be pursued, where optical access to scalar and velocity measurements allow the investigation of the turbulence-chemistry interactions, which will be benefical for LES model development.  Furthermore, since soot particles are strong radiators in the flame, radiative heat transfer between soot and the ambient gas needs to be quantified using more accurate models than the optically thin model for some conditions.  Coupling the radiation model with the current combustion and soot models can be challenging.  Again, low soot loading laminar flames with well-characterized flow and temperature fields can initiate the investigation, and complexities from large hydrocarbon fuels, flow configuration, and turbulence can be brought in gradually. 

\end{itemize}
